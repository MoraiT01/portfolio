\documentclass[aspectratio=169,xcolor=dvipsnames]{beamer}
\usetheme{SimplePlus}
\usepackage[german]{babel}
\usepackage[babel=true]{csquotes}
\usepackage{hyperref}
\usepackage{graphicx}
\usepackage{booktabs}

\title[HAnS Epics]{HAnS Epics}
\subtitle{und Storys}

\institute[THI, THN, HSH]
{
    Technische Hochschule Ingolstadt,
    Technische Hochschule Nürnberg,
    Hochschule Hof
}
\date{07.07.2022}
%----------------------

\begin{document}

\begin{frame}
    \titlepage
\end{frame}

\begin{frame}{Übersicht}
    \tableofcontents
\end{frame}

%------------------------------------------------

%------------------------------------------------
\section{Notes from meeting 07-07-22}
%------------------------------------------------

\begin{frame}{Topics of interest\\
Mentioned during meeting on 07-07-22}
    \begin{itemize}
        \item 
        Thomas: OCR zur Extraktion von Text zur ASR Verbesserung
        \item
        Fabian: Speaker Diarization for Anonymization, Text summarization und Topic segmentation
        \item
        Christopher: ASR Verbesserung durch Lippenlesung
        \item
        Lia: Speaker Diarization, eigenes (E2E) ASR System
        
    \end{itemize}

\end{frame}

%------------------------------------------------
\section{Epic 1: Audio Mining \& Segmentation}
%------------------------------------------------

\begin{frame}{Epic 1: Audio Mining \& Segmentation\\
Story 1: Segmentation \& Alignment}
    Task 1: Segmentation
    \begin{itemize}
        \item 
        Segmentierung des Audiosignals auf Satzebene mith. akustischer Features
        
        \item
        Satzsegmente besser geeignet für ASR und spätere Verarbeitung in HAnS
        
    \end{itemize}

    Task 2: Alignment
    \begin{itemize}
        \item
        Alignment von Audio und Transkript auf Wortebene mith. phonemischer Features
        
        \item
        Wichtig für Begriffsuche im Video und Adaption (Aussprache von Fachbegriffen)
        
    \end{itemize}

\vfill    
\scriptsize \textbf{AP 2.1.1} ASR Grundsystem und Anpassung an Lehrmaterial\\
\textbf{AP 2.2.1} Audio Mining, zeitsynchroner Lattices \& Konfidenzskalierung 
\end{frame}

%------------------------------------------------

\begin{frame}{Epic 1: Audio Mining \& Segmentation\\
Story 2: Topic Segmentation}

    \begin{itemize}
        \item 
        Unterteilung der Vorlesungsvideos in Themenblöcke und Benennung der Subthemen
        
        \item
        Unterteilung der Vorlesung in Themenblöcke erleichtert Weiterverarbeitung im HAnS-Kontext und erhöht Nutzerfreundlichkeit
        
        \item
        Unsupervised \& Supervised Ansätze zur Topic Segmentation und Topic Extraction
        
    \end{itemize}

\vfill
\scriptsize \textbf{AP 2.2.4} Topic Modeling mit User-Interaktion
\end{frame}

%------------------------------------------------

\begin{frame}{Epic 1: Audio Mining \& Segmentation\\
Story 3: Audio Mining}
    \begin{itemize}
        \item 
        Detektieren von Störquellen/Soundproblemen (und anderen Sound Events) in der Audiosequenz erleichtert nachfolgende Bearbeitung in HAnS
        
        \item
        Nimmt Lehrenden Arbeit ab bei der Durchsicht der Aufzeichnung - Anreiz zur Verwendung von HAnS
        
    \end{itemize}

\vfill    
\scriptsize  \textbf{AP 2.1.1} ASR Grundsystem und Anpassung an Lehrmaterial\\
\textbf{AP 2.2.1} Audio Mining, zeitsynchroner Lattices \& Konfidenzskalierung
\end{frame}

%------------------------------------------------
\section{Epic 2: Adaption \& Anonymisierung}
%------------------------------------------------

\begin{frame}{Epic 2: Adaption \& Anonymisierung \\ Story 1: Textbasierte Adaption}
    \begin{itemize}
        \item Domain Adaptation der Language Models mit Transkripten und von OCR
        \item Lernen und Erkennen von Fachbegriffen für ASR (Lattice Rescoring), Topic Modeling, usw.\
        \begin{itemize}
            \item[$\rightarrow$] Generierung von domänespezifischen Vokabularen
        \end{itemize}
        \item Mehrsprachige Synonyme von Fachbegriffen sollten als solche erkannt werden
        \begin{itemize}
            \item[$\rightarrow$] Auch evtl. als Schnittstelle zwischen deutschem und englischem Lehrmaterial
        \end{itemize}
        \item Kontinuierliche Verbesserung nach dem Deployment (online learning) und Möglichkeit zur Ergänzung mit neuen Domains
    \end{itemize}

\vfill    
\scriptsize \textbf{AP 2.1.1} Anpassung an Lehrmaterial, \textbf{2.1.3} Domain-Adaptation mit OCR\\
\textbf{AP 2.2.1} zeitsynchrone Lattices, \textbf{2.2.2} Erkennung \& Adaption von Fachbegriffen, \textbf{2.2.3} Lernen von Fachausdrücken    
    
\end{frame}

%------------------------------------------------

\begin{frame}{Epic 2: Adaption \& Anonymisierung \\ Story 2: Akustische Adaption}
    \begin{itemize}
        \item Dynamische Sprecheradaption für das akustische Modell
        \begin{itemize}
            \item[$\rightarrow$] Auch nach Deployment muss an neue Sprecher adaptiert werden    
        \end{itemize}
        \item Anpassung an verschiedene Räume und Aufnahmegeräte
        \item Konfidenzskalierung für die manuelle Korrektur möglicher Transkriptionsfehler
        \item Emphasis Detection zur Unterstützung der Fachbegrifferkennung und als Metainformation im Transkript
        \begin{itemize}
            \item[$\rightarrow$] Kann in die UI integriert und von der Suchfunktion verwendet werden
        \end{itemize}
    \end{itemize}
\vfill    
\scriptsize \textbf{AP 2.1.4} Sprecheradaption\\
\textbf{AP 2.2.1} Audiomining \& Konfidenzskalierung

\end{frame}

%------------------------------------------------

\begin{frame}{Epic 2: Adaption \& Anonymisierung \\ Story 3: Speaker Diarization und Anonymisierung}
    \begin{itemize}
        \item Speaker Diarization zur Erkennung und Anonymisierung von Abschnitten, in denen ein Student spricht
        \item Hauptsprechererkennung für die Sprecheradaption
        \item TTS-System von Transkription anonymisierter Abschnitte für Accessibility
        \begin{itemize}
            \item[$\rightarrow$] Könnte auch für den KI-Tutor verwendet werden
        \end{itemize}
    \end{itemize}
\vfill    
\scriptsize\textbf{AP 2.1.5} Sprechererkennung/Anonymisierung der Sprache von Studenten
\end{frame}


%------------------------------------------------
\section{Epic 3: Test \& Evaluation}
%------------------------------------------------

\begin{frame}{Epic 3: Test \& Evaluation \\ Story 1: Aufnahmen}
\begin{itemize}
    \item 
%    Motivation:
    Dozierende benutzen unterschiedliche Mikrofone in unterschiedlichen Räumen
    
    \item 
%    Problem:
    Spracherkennung \& Sprecheradaption sollen trotzdem funktionieren
    
    \item 
%    Lösung: 
    Kontrollierte Aufnahmen in verschiedenen Umgebungen

\end{itemize}
    
\vfill    
\scriptsize\textbf{AP 2.1.1} Grundsystem\\
\textbf{AP 2.1.4} Sprecheradaption
    
    
\end{frame}

%------------------------------------------------

\begin{frame}{Epic 3: Test \& Evaluation \\ Story 2: Analyse}

\begin{itemize}
    \item 
 %   Motivation: 
    Viele Hochschulen mit verschiedenen Fachbereichen benutzen HAnS
    
    \item 
%    Problem: 
    Akzente \& Fachbegriffe sollen erkannt werden
    
    \item 
%    Lösung: 
    Bias-Analyse | Wo braucht die Erkennung noch Verbesserung?
    
\end{itemize}
\vfill    
\scriptsize\textbf{AP 2.1.1} Grundsystem\\
\textbf{AP 2.1.4} Sprecheradaption\\
\textbf{AP 2.2.2} Fachbegriffe
\end{frame}


%------------------------------------------------



\end{document}